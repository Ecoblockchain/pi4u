\documentclass{article}
\usepackage{amsmath}
\usepackage{stmaryrd}
\newcommand{\interp}[1]{\llbracket #1 \rrbracket}
\newcommand{\maps}{\colon}
\newcommand{\gram}[1]{\mbox{\tt <#1>}}

\title{Logic as a distributive law}
\author{
Michael Stay\\
  {Pyrofex Corp.}\\
  {\fontsize{8}{8}\selectfont stay@pyrofex.net}
\and
L.G. Meredith\\
  {Biosimilarity, LLC}\\
  {\fontsize{8}{8}\selectfont lgreg.meredith@biosimilarity.com}
}
\begin{document}
\maketitle
\begin{abstract}
\noindent
We present an algorithm that takes a formal presentation of a virtual machine and outputs a spatial-behavioral type system.  The presentation consists of a context-free grammar for the state of the machine together with a set of rewrite rules describing how the state changes---in category-theoretic terminology, a Lawvere 2-theory.  The types are ``types \`a la Curry'' or in Reynolds' terminology, ``extrinsic'' types: they are propositions about machine state and behavior.  We interpret a formula as a collection of terms of the grammar that satisfy the formula, and we interpret proofs as morphisms of collections.  A distributive law forms the core of this interpretation.

In the simplest case, formulae and proofs are given by the sum of the 2-theory $T$ for the machine with a 2-theory $C$ for a collection.  Since each 2-theory corresponds to a finitary 2-monad, the interpretation of formulae is a natural transformation from $T+C$ to $CT$ built from the unit and multiplication transformations and a distributive law natural transformation $\delta\maps TC \Rightarrow CT$.

We present several examples of the logics so generated and some interesting formulae in those logics.
\end{abstract}

\section{Introduction}
[[ Expand this. ]]
We present an algorithm that takes a formal presentation of a virtual machine and outputs a spatial-behavioral type system.  The presentation consists of a context-free grammar for the state of the machine together with a set of rewrite rules describing how the state changes---in category-theoretic terminology, a Lawvere 2-theory.  The types are ``types \`a la Curry'' or in Reynolds' terminology, ``extrinsic'' types: they are propositions about machine state and behavior.  We interpret a formula as a collection of terms of the grammar that satisfy the formula, and we interpret proofs as morphisms of collections.  A distributive law forms the core of this interpretation.

In the simplest case, formulae and proofs are given by the sum of the 2-theory $T$ for the machine with a 2-theory $C$ for a collection.  Since each 2-theory corresponds to a finitary 2-monad, the interpretation of formulae is a natural transformation from $T+C$ to $CT$ built from the unit and multiplication transformations and a distributive law natural transformation $\delta\maps TC \Rightarrow CT$.

We present several examples of the logics so generated and some interesting formulae in those logics.

\subsection{A motivating example}

Before considering the full 2-categorical machinery, it will be helpful to consider a 1-categorical example.

Consider this BNF grammar on a set $G$ of generators.
\begin{center}
  \tt
  \begin{tabular}{rll}
    <monoid>[G] ::= & <G> &|\\
    & 1 &|\\
    & ( <monoid>[G] * <monoid>[G] )
  \end{tabular}
\end{center}
The grammar describes a set of binary trees whose leaves are labeled either by 1 or the elements of $G.$  In order to describe the free monoid on $G,$ we need to impose associativity and unit laws:
\begin{center}
  \tt
  \begin{tabular}{rl}
    For all x, y, z in <monoid>[G], & ((x * y) * z) = (x * (y * z))\\
    For all x in <monoid>[G], & (x * 1) = x\\
    For all x in <monoid>[G], & (1 * x) = x
  \end{tabular}
\end{center}
The grammar and the equations taken together are a presentation of the monad for the free monoid on the set $G$.

To construct a logic where a formula, say $\phi$, denotes the
collection of monoid expressions in {\tt <monoid>[G]} satisfying $\phi$, 
we need to specify what \emph{kind} of collection, because different kinds
of collections have different constraints that affect how we construct them.
For instance, lists are sensitive to order where sets are not. Bags are
sensitive to multiplicity where sets are not. In this sense, sets are
the most insensitive monadically presented notion of collection, and
thus are ideal for erasing detail about collectivity that might
distract when bootstrapping an understanding of what it means to
collect or gather things together.

Given a set $G$ of generators, we can describe sets of elements from $G$ with the grammar
\begin{center}
  \tt
  \begin{tabular}{rll}
    <set>[G] ::= & <G> &|\\
    & T &| \\
    & F &| \\
    & (<set>[G] $\land$ <set>[G]) &| \\
    & (<set>[G] $\lor$ <set>[G]) &| \\
    & $\neg$ <set>[G]
  \end{tabular}
\end{center}
equipped with equations for associativity, unit laws, and commutativity of $\land$ and $\lor$, idempotence of $\neg,$ and de Morgan's laws.  The grammar and equations are a presentation of the monad for the free Boolean algebra on the set $G.$

The sum of the two monads is a new monad whose terms are our formulae.  The sum is presented by taking the union of the grammars and the union of the equations.
\begin{center}
  \tt
  \begin{tabular}{rll}
    <formula>[G] ::= & <G> &| \\
    & 1 &| \\
    & ( <formula>[G] * <formula>[G] ) &|\\
    & T &|\\
    & F &| \\
    & (<formula>[G] $\land$ <formula>[G]) &| \\
    & (<formula>[G] $\lor$ <formula>[G]) &| \\
    & $\neg$ <formula>[G]
  \end{tabular}
\end{center}
Since the sum of two monads is the free product of the two, a general formula will be a term in an alternating composition of the two monads.  For example, suppose that $a, b, c, d, e \in G;$ then one formula is 
\begin{center}
  \tt ((a $\lor$ (b * e)) * (c $\lor$ d)),
\end{center}
which is a term in {\tt <monoid>[<set>[<monoid>[G]]]}.  

The interpretation of this formula should be the set of monoid words
\[ \{ ac, ad, bec, bed \}, \]
or in other words, the term
\begin{center}
  \tt (a * c) $\lor$ (a * d) $\lor$ (b * e * c) $\lor$ (b * e * d)
\end{center}
in {\tt <set>[<monoid>[G]]}.  

In order to move all the uses of {\tt <monoid>} to the right of the uses of {\tt <set>}, we need a distributive law natural transformation
\[ \delta\maps \gram{monoid}\circ\gram{set} \Rightarrow \gram{set}\circ\gram{monoid}. \]
Given $\delta$ and the monad units and multiplications, we can define an interpretation natural transformation
\[ \interp{-}\maps \gram{formula}\Rightarrow\gram{set}\circ\gram{monoid} \]
in the obvious way:
\begin{align*}
  \interp{T}_G &= \mbox{\tt monoid[G]}\\
  \interp{F}_G &= \emptyset\\
  \interp{{A}\lor{B}}_G &= \interp{A}_G \cup \interp{B}_G\\
  \interp{{A}\land{B}}_G &= \interp{A}_G \cap \interp{B}_G\\
  \interp{\neg A}_G &= \mbox{\tt monoid[G]} - \interp{A}_G\\
  \interp{m \in \mbox{\tt monoid[G]}}_G &= \{m\}
\end{align*}

Even in this simple example, we have nontrivial formulae; for example,
\[ prime = \neg 1 \land \neg(\neg 1 * \neg 1) \]
is a 1-line formula for primality.  It is easy to verify that $\interp{prime}_G = G.$

\section{Related work}
\end{document}