% ------------------------------------------------------------------------
% AMS-LaTeX Paper ********************************************************
% ------------------------------------------------------------------------
% Submitted:      Dec 15 2003
% Final Version:  
% Accepted:       
% ------------------------------------------------------------------------
% This is a journal top-matter template file for use with AMS-LaTeX.
%%%%%%%%%%%%%%%%%%%%%%%%%%%%%%%%%%%%%%%%%%%%%%%%%%%%%%%%%%%%%%%%%%%%%%%%%%

%\documentclass{tran-l}
%\documentclass[twocolumn]{amsart}
%\documentclass[]{amsart}
%\documentclass[]{sig-alternate}
\documentclass[]{acm_proc_article-sp}
%\documentclass[]{llncs}


%\documentclass[]{prentcsmacro}

%\usepackage[active]{srcltx} % SRC Specials for DVI Searching
\usepackage{url}
\usepackage[pdf]{pstricks}
\usepackage{pstricks-add, pst-grad, pst-plot}
\usepackage[tiling]{pst-fill}
\psset{linewidth=0.3pt,dimen=middle}
\psset{xunit=.70cm,yunit=0.70cm}
\psset{angleA=-90,angleB=90,ArrowInside=->,arrowscale=2}


% From Allen's stable.
\usepackage{bigpage}
\usepackage{bcprules}
%\usepackage{code}
\usepackage{mathpartir}
\usepackage{listings}
\usepackage{mathtools}
%\usepackage[fleqn]{amsmath}
\usepackage{amsfonts}
\usepackage{latexsym}
\usepackage{amssymb}
\usepackage{caption}
%\usepackage{multicol}

% Math
\newcommand{\maps}{\colon}
\newcommand{\NN}{\mathbb{N}}
% Double brackets
\newcommand{\ldb}{[\![}
\newcommand{\rdb}{]\!]}
\newcommand{\ldrb}{(\!(}
\newcommand{\rdrb}{)\!)}
\newcommand{\lliftb}{\langle\!|}
\newcommand{\rliftb}{|\!\rangle}
% \newcommand{\lpquote}{\langle}
% \newcommand{\rpquote}{\rangle}
% \newcommand{\lpquote}{\lceil}
% \newcommand{\rpquote}{\rceil}
\newcommand{\lpquote}{\ulcorner}
\newcommand{\rpquote}{\urcorner}
\newcommand{\newkw}{\nu}

% SYNTAX
\newcommand{\id}[1]{\texttt{#1}}
\newcommand{\none}{\emptyset}
\newcommand{\eps}{\epsilon}
\newcommand{\set}[1]{\{#1\}}
\newcommand{\rep}[2]{\id{\{$#1$,$#2$\}}}
\newcommand{\elt}[2]{\id{$#1$[$#2$]}}
\newcommand{\infinity}{$\infty$}

\newcommand{\pzero}{\mathbin{0}}
\newcommand{\seq}{\mathbin{\id{,}}}
\newcommand{\all}{\mathbin{\id{\&}}}
\newcommand{\choice}{\mathbin{\id{|}}}
\newcommand{\altern}{\mathbin{\id{+}}}
\newcommand{\juxtap}{\mathbin{\id{|}}}
%\newcommand{\concat}{\mathbin{.}}
\newcommand{\concat}{\Rightarrow}
\newcommand{\punify}{\mathbin{\id{:=:}}}
\newcommand{\fuse}{\mathbin{\id{=}}}
\newcommand{\scong}{\mathbin{\equiv}}
\newcommand{\nameeq}{\mathbin{\equiv_N}}
\newcommand{\alphaeq}{\mathbin{\equiv_{\alpha}}}
\newcommand{\names}[1]{\mathbin{\mathcal{N}(#1)}}
\newcommand{\freenames}[1]{\mathbin{\mathcal{FN}(#1)}}
\newcommand{\boundnames}[1]{\mathbin{\mathcal{BN}(#1)}}
%\newcommand{\lift}[2]{\texttt{lift} \; #1 \concat #2}
\newcommand{\binpar}[2]{#1 \juxtap #2}
\newcommand{\outputp}[2]{#1 ! ( * #2 )}
\newcommand{\prefix}[3]{#1 ? ( #2 ) \concat #3}
\newcommand{\lift}[2]{#1 ! ( #2 )}
%\newcommand{\quotep}[1]{\lpquote #1 \rpquote}
\newcommand{\quotep}[1]{@#1}
\newcommand{\dropn}[1]{*#1}

\newcommand{\newp}[2]{\id{(}\newkw \; #1 \id{)} #2}
\newcommand{\bangp}[1]{\int #1}
\newcommand{\xbangp}[2]{\int_{#2} #1}
\newcommand{\bangxp}[2]{\int^{#2} #1}

\newcommand{\substp}[2]{\id{\{} \quotep{#1} / \quotep{#2} \id{\}}}
\newcommand{\substn}[2]{\id{\{} #1 / #2 \id{\}}}

\newcommand{\psubstp}[2]{\widehat{\substp{#1}{#2}}}
\newcommand{\psubstn}[2]{\widehat{\substn{#1}{#2}}}

\newcommand{\applyp}[2]{#1 \langle #2 \rangle}
\newcommand{\absp}[2]{\id{(} #1 \id{)} #2}

\newcommand{\transitions}[3]{\mathbin{#1 \stackrel{#2}{\longrightarrow} #3}}
\newcommand{\meaningof}[1]{\ldb #1 \rdb}
\newcommand{\pmeaningof}[1]{\ldb #1 \rdb}
\newcommand{\nmeaningof}[1]{\ldrb #1 \rdrb}

\newcommand{\Proc}{\mathbin{Proc}}
\newcommand{\QProc}{\quotep{\mathbin{Proc}}}

\newcommand{\entailm}{\mathbin{\vdash_{\mathfrak m}}} %matching
\newcommand{\entailp}{\mathbin{\vdash_{\mathfrak p}}} %behavioral
\newcommand{\entailv}{\mathbin{\vdash_{\mathfrak v}}} %validation
\newcommand{\congd}{\mathbin{\equiv_{\mathfrak d}}}
\newcommand{\congs}{\mathbin{\equiv_{\mathfrak s}}}
\newcommand{\congp}{\mathbin{\equiv_{\mathfrak p}}}
%\newcommand{\defneqls}{:\!=}
\newcommand{\defneqls}{\coloneqq}
%\newcommand{\logequiv}{\mathbin{\leftrightarrow}}

\newcommand{\barb}[2]{\mathbin{#1 \downarrow_{#2}}}
\newcommand{\dbarb}[2]{\mathbin{#1 \Downarrow_{#2}}}

% From pi-duce paper
\renewcommand{\red}{\rightarrow}
\newcommand{\wred}{\Rightarrow}
\newcommand{\redhat}{\hat{\longrightarrow}}
\newcommand{\lred}[1]{\stackrel{#1}{\longrightarrow}} %transitions
\newcommand{\wlred}[1]{\stackrel{#1}{\Longrightarrow}}

\newcommand{\opm}[2]{\overline{#1} [ #2 ]} % monadic
\newcommand{\ipm}[2]{{#1} ( #2 )} 
\newcommand{\ipmv}[2]{{#1} ( #2 )} % monadic
\newcommand{\parop}{\;|\;}    % parallel operator
\newcommand{\patmatch}[3]{#2 \in #3 \Rightarrow #1}
\newcommand{\sdot}{\, . \,}    % Space around '.'
\newcommand{\bang}{!\,}
%\newcommand{\fuse}[1]{\langle #1 \rangle}    
\newcommand{\fusion}[2]{#1 = #2} % fusion prefix/action
\newcommand{\rec}[2]{\mbox{\textsf{rec}} \, #1. \, #2}
\newcommand{\match}[2]{\mbox{\textsf{match}} \; #1 \; \mbox{\textsf{with}} \; #2}
\newcommand{\sep}{:}
\newcommand{\val}[2]{\mbox{\textsf{val}} \; #1 \; \mbox{\textsf{as}} \; #2}

\newcommand{\rel}[1]{\;{\mathcal #1}\;} %relation
\newcommand{\bisim}{\stackrel{.}{\sim}_b} %bisimilar
\newcommand{\wb}{\approx_b} %weak bisimilar
\newcommand{\bbisim}{\stackrel{\centerdot}{\sim}} %barbed bisimilar
\newcommand{\wbbisim}{\stackrel{\centerdot}{\approx}} %weak barbed bisimilar
\newcommand{\wbbisimsem}{\approx} %weak barbed bisimilar
\newcommand{\bxless}{\lesssim}  %expansion less (amssymb required)
\newcommand{\bxgtr}{\gtrsim}  %expansion greater (amssymb required)
\newcommand{\beq}{\sim}    %barbed congruent
\newcommand{\fwbeq}{\stackrel{\circ}{\approx}}  %weak barbed congruent
\newcommand{\wbeq}{\approx}  %weak barbed congruent
\newcommand{\sheq}{\simeq}  %symbolic hypereq
\newcommand{\wbc}{\approx_{cb}}

% End piduce contribution

% rho logic

\newcommand{\ptrue}{\mathbin{true}}
\newcommand{\psatisfies}[2]{#1 \models #2}
\newcommand{\pdropf}[1]{\rpquote #1 \lpquote}
\newcommand{\pquotep}[1]{\lpquote #1 \rpquote}
\newcommand{\plift}[2]{#1 ! ( #2 )}
\newcommand{\pprefix}[3]{\langle #1 ? #2 \rangle #3}
\newcommand{\pgfp}[2]{\textsf{rec} \; #1 \mathbin{.} #2}
\newcommand{\pquant}[3]{\forall #1 \mathbin{:} #2 \mathbin{.} #3}
\newcommand{\pquantuntyped}[2]{\forall #1 \mathbin{.} #2}
\newcommand{\riff}{\Leftrightarrow}

\newcommand{\PFormula}{\mathbin{PForm}}
\newcommand{\QFormula}{\mathbin{QForm}}
\newcommand{\PropVar}{\mathbin{\mathcal{V}}}

\newcommand{\typedby}{\mathbin{\:\colon}}
\newcommand{\mixedgroup}[1]{\id{mixed($#1$)}}
\newcommand{\cast}[2]{\id{CAST AS} \; #1 \; (#2)}
\newcommand{\bslsh}{\mathbin{\id{\\}}}
\newcommand{\bslshslsh}{\mathbin{\id{\\\\}}}
\newcommand{\fslsh}{\mathbin{\id{/}}}
\newcommand{\fslshslsh}{\mathbin{\id{//}}}
\newcommand{\bb}[1]{\mbox{#1}}
\newcommand{\bc}{\mathbin{\mathbf{::=}}}
\newcommand{\bm}{\mathbin{\mathbf\mid}}
\newcommand{\be}{\mathbin{=}}
\newcommand{\bd}{\mathbin{\buildrel {\rm \scriptscriptstyle def} \over \be}}
\newcommand{\ctcategory}[1]{\mbox{\bf #1}}

%GRAMMAR
\newlength{\ltext}
\newlength{\lmath}
\newlength{\cmath}
\newlength{\rmath}
\newlength{\rtext}

\settowidth{\ltext}{complex type name}
\settowidth{\lmath}{$xxx$}
\settowidth{\cmath}{$::=$}
\settowidth{\rmath}{\id{attributeGroup}}
\settowidth{\rtext}{repetition of $g$ between $m$ and $n$ times}

\newenvironment{grammar}{
  \[
  \begin{array}{l@{\quad}rcl@{\quad}l}
  \hspace{\ltext} & \hspace{\lmath} & \hspace{\cmath} & \hspace{\rmath} & \hspace{\rtext} \\
}{
  \end{array}\]
}

% Over-full v-boxes on even pages are due to the \v{c} in author's name
\vfuzz2pt % Don't report over-full v-boxes if over-edge is small

% THEOREM Environments ---------------------------------------------------
 \newtheorem{thm}{Theorem}[subsection]
 \newtheorem{cor}[thm]{Corollary}
 \newtheorem{lem}[thm]{Lemma}
 \newtheorem{prop}[thm]{Proposition}
% \theoremstyle{definition}
 \newtheorem{defn}[thm]{Definition}
% \theoremstyle{remark}
 \newtheorem{rem}[thm]{Remark}
 \newtheorem{example}[thm]{Example}
 \numberwithin{equation}{subsection}
% MATH -------------------------------------------------------------------
 \DeclareMathOperator{\RE}{Re}
 \DeclareMathOperator{\IM}{Im}
 \DeclareMathOperator{\ess}{ess}
 \newcommand{\veps}{\varepsilon}
 \newcommand{\To}{\longrightarrow}
 \newcommand{\h}{\mathcal{H}}
 \newcommand{\s}{\mathcal{S}}
 \newcommand{\A}{\mathcal{A}}
 \newcommand{\J}{\mathcal{J}}
 \newcommand{\M}{\mathcal{M}}
 \newcommand{\W}{\mathcal{W}}
 \newcommand{\X}{\mathcal{X}}
 \newcommand{\BOP}{\mathbf{B}}
 \newcommand{\BH}{\mathbf{B}(\mathcal{H})}
 \newcommand{\KH}{\mathcal{K}(\mathcal{H})}
 \newcommand{\Real}{\mathbb{R}}
 \newcommand{\Complex}{\mathbb{C}}
 \newcommand{\Field}{\mathbb{F}}
 \newcommand{\RPlus}{\Real^{+}}
 \renewcommand{\Polar}{\mathcal{P}_{\s}}
 \newcommand{\Poly}{\mathcal{P}(E)}
 \newcommand{\EssD}{\mathcal{D}}
 \newcommand{\Lom}{\mathcal{L}}
 \newcommand{\States}{\mathcal{T}}
 \newcommand{\abs}[1]{\left\vert#1\right\vert}
% \newcommand{\set}[1]{\left\{#1\right\}}
%\newcommand{\seq}[1]{\left<#1\right>}
 \newcommand{\norm}[1]{\left\Vert#1\right\Vert}
 \newcommand{\essnorm}[1]{\norm{#1}_{\ess}}

%%% NAMES
\newcommand{\Names}{{\mathcal N}}
\newcommand{\Channels}{{\sf X}}
\newcommand{\Variables}{{\mathcal V}}
\newcommand{\Enames}{{\mathcal E}}
\newcommand{\Nonterminals}{{\mathcal S}}
\newcommand{\Pnames}{{\mathcal P}}
\newcommand{\Dnames}{{\mathcal D}}
\newcommand{\Types}{{\mathcal T}}

\newcommand{\fcalc}{fusion calculus}
\newcommand{\xcalc}{${\mathfrak x}$-calculus}
\newcommand{\lcalc}{$\lambda$-calculus}
\newcommand{\pic}{$\pi$-calculus}
\newcommand{\rhoc}{${\textsc{rho}}$-calculus}
\newcommand{\hcalc}{highwire calculus}
\newcommand{\dcalc}{data calculus}
%XML should be all caps, not small caps. --cb
%\newcommand{\xml}{\textsc{xml}}
\newcommand{\xml}{XML} 

\newcommand{\papertitle}{Higher category models of the pi-calculus}
% use static date to preserve date of actual publication
 \newcommand{\paperversion}{Draft Version 0.1 - Jan 7, 2015}

\newenvironment{toc}
{
\begin{list}{}{
   \setlength{\leftmargin}{0.4in}
   \setlength{\rightmargin}{0.6in}
   \setlength{\parskip}{0pt}
 } \item }
{\end{list}}

\newenvironment{narrow}
{
\begin{list}{}{
   \setlength{\leftmargin}{0.4in}
   \setlength{\rightmargin}{0.6in}
 } \item }
{\end{list}}

%%% ----------------------------------------------------------------------

%\title{\huge{\papertitle}}
\title{\papertitle}

%\numberofauthors{3}
\author{
Michael Stay\\
  \affaddr{Google}\\
  \email{\fontsize{8}{8}\selectfont stay@google.com}
\and
L.G. Meredith\\
  \affaddr{Biosimilarity, LLC}\\
  \email{\fontsize{8}{8}\selectfont lgreg.meredith@biosimilarity.com}
}

%\address{Systems Biology, Harvard Medical School, Boston, Massachussetts, USA}

%\email{lg_meredith@hms.harvard.edu}

%\thanks{This work was completed during a visiting appointment at the Department of Systems Biology, Harvard Medical School.}

%\subjclass{Primary 47A15; Secondary 46A32, 47D20}

%\date{April 6, 2002.}

%\dedicatory{}

%\commby{Daniel J. Rudolph}

%%% ----------------------------------------------------------------------

\begin{document}
%\lstset{language=erlang}
\lstset{language=}

%These margin values appear to be relative to the bigpage package settings. --cb
\setlength{\topmargin}{0in}
\setlength{\textheight}{8.5in}
\setlength{\parskip}{6pt}

\keywords{ higher category theory, concurrency, message-passing, types, Curry-Howard }

\begin{abstract}
\normalsize{ 

  We present a realizability interpretation of logics as distributive
  laws over monads. Roughly speaking, if formulae are to denote the
  collection of individual computations that satisfy them, we must
  provide three data to specify a logic: the language of individual
  computations, which is typically captured by the algebras of a
  monad, say T, for terms; the notion of collection, such as set, or
  bag, or list, or ... used to gather the terms that satisfy a
  particular formulae, which is typically captured by some monad, C,
  for collection. Finally, given a distributive law, $l : TC
  \rightarrow CT$, we find a logic whose formulae are isomorphic to
  $TC$, the semantics of which are given in $CT$ by $l$. The logic
  will enjoy three kinds of formulae: formulae corresponding to the
  structure of collections unconstrained by either requirements on
  term structure, or requirements on the evolution of computation;
  formulae corresponding to the expression of constraints on term
  structure; formulae corresponding to the expression of constraints
  on computational evolution. We present several examples of the
  logics so generated, including logics for monoids, the lambda
  calculus, the {\pic}, as well as formulae illustrating the
  expressive power of the logics so generated.

}

\end{abstract}

% \noindent
% {\large \textbf{Submission to arXiv}}\\
% \rule{6.25in}{0.75pt}\\\\\\

%%% ----------------------------------------------------------------------
\maketitle
%%% ----------------------------------------------------------------------

% \begin{center}
% \paperversion\\
% \end{center}

% \begin{toc}
% \tableofcontents
% \end{toc}

% \newpage
% ------------------------------------------------------------------------

\section{Introduction}

So far, the picture looks like this.  A term calculus has three parts:
\begin{itemize}
  \item a BNF grammar describing the syntax of the terms
  \item an equivalence, called "structural equivalence" that says what variations in the syntax we ignore; for example, changing the name of a dummy variable in a lambda term ("alpha equivalence"), or the commutativity of concurrent processes.
  \item a reduction relation that captures what we mean by the process of computation; for example, beta reduction in lambda calculus or synchronization on a name in {\pic}.
  \item a congruence on reduction that tells what computational processes we consider equivalent.
\end{itemize}


We encode this information as a Lawvere 2-theory Th.  We have
\begin{itemize}
  \item one sort for the terms,
  \item a function symbol for each term constructor,
  \item rewrites for both structural equivalence and the rewrite relation, and
  \item relations on rewrites that say that structural equivalence rewrites are the same as the identity (e.g. we treat a lambda term as independent of the choice of dummy variable).
\end{itemize}

If reductions only occur in certain contexts, then we add function
symbols to reify the contexts and use those in the rewrite rules.

If reductions only occur in certain contexts, then we add function
symbols to reify the contexts and use those in the rewrite rules.
We take models in Cat; the 2-category of functors, transformations and
modifications is equivalent to the 2-category Model(Th) of models of
the calculus.

There's a forgetful functor from Model(Th) to Cat with a left adjoint.
Together they form a monad on Cat.


If we assume that our 2-categories have coproducts, then the sum of
the left hand sides of the term constructors gives an endofunctor on
the theory.  LHS is a monad on Model(Th) by construction.


The sum of the term constructors themselves, together with T, form an
algebra.  The free model on no generators gives the initial algebra in
the model 2-category.


Next take another 2-theory of a data type, i.e. no unidirectional
2-morphisms (in the simple version).  This time we interpret it in
Model(Th).  Call the 2-category of functors, transformations, and
modifications Collect(Th).  Then there's a forgetful functor from
Collect(Th) to Model(Th) with a left adjoint.  Together they form the
free collection monad on Model(Th).


Next we need a "distributive law" LHS(C(T)) -> C(LHS(T)) in Model(Th).


We define formulae with operations from C, spatial-behavioral types
from LHS \& 2-morphisms of Th.  When considering normalization of data
types, 2-morphisms from the data type will probably show up in the
modalities.

Need a proper accounting of the interpretation functor from formulae
to collections of terms, and how modalities give properties like
closedness.

\subsubsection{Related work}

%Montanari, et al have considered double category models of the {\pic}.

TBD

\subsubsection{Organization of the rest of the paper}

TBD

%%% ----------------------------------------------------------------------

\section{Some motivating examples}

In this section we motivate the construction by way of a few examples
familiar both to computer scientists and category theorists.

\subsection{A logic for monoids}

The notion of monoid captures fundamental, yet minimal structure in a
wide variety of computational phenomena. Data structures, such as
lists, enjoy monoidal structure; arithmetic phenomena, such as the
natural numbers under addition, enjoy monoidal structure; executive or
control flow phenomena, such as parallel composition of communicating
processes enjoy monoidal structure. For this reason, they are widely
studied and relatively well understood by the computing community.

They also play a fundamental role in category theory. If category
theory is a theory of composition, and monoids expose one of the most
basic notions of composition, then categorical accounts of monoids and
monoidal phenomena say as much about category theory as the other way
around. Thus, the fact that the free monoid, $T[G]$, on a set of $n$
generators, say $G = \{ g_i : 1 \leq i \leq n \}$, is a monad and that
monads are monoid objects in a category of endofunctors says a great
deal about the notions of composition at play in category theory. In
particular, it illustrates the process of categorification raises the
level of abstraction, revealing fundamental patterns in operators in a
much broader range of phenomena.

Where computer science and category theory come together to study
these phenomena we see a new kind of language arise, one with the
computer scientist's emphasis on effective presentations, yet with the
category theorists view to higher and higher levels of abstraction and
expressiveness. In this language, we can present the basic data of the
free monoid on a set of $n$ generators in a manner that supports both
computation and higher levels of abstraction.

\label{syntax}
\begin{grammar}
{T[G]} \bc e & \mbox{identity} \\
       \;\;\; \bm \; g_i & \mbox{generator in G} \\
       \;\;\; \bm \; T[G] * T[G] & \mbox{composition} \\
\end{grammar}

This sort of notation follows the computer scientist's use of grammars
to present abstract syntax. Such a presentation provides a natural and
compact way to present and study algebraic and computational
structure. It does so in two stages. Firstly, it recursively describes
a set of purely syntactic entities, monoid expressions, if you will,
in terms of a set of building blocks, namely the identity, $e$, and
the generators $g_i \in G$. Secondly, it describes a structural
equivalence relation, $\scong$, as the smallest equivalence relation
on $T[G]$ satisfying

\begin{equation*}
  \begin{aligned}
    t * e \scong t \scong e * t \\
    t_1 * ( t_2 * t_3 ) \scong ( t_1 * t_2 ) * t_3
  \end{aligned}
\end{equation*}

This presentation should also be somewhat familiar to the category
theorist who is well acquainted with generators and relations style
presentations of algebras. The notion of grammar generalizes the
notion of generators, while the notion of structural equivalence
generalizes the notion of relations.

At a higher level of abstraction, the recursive specification of
$T[G]$ makes it clear that $T[G]$ computes the transitive closure of
some operator $T$ on $G$. Recalling that closure operators are monads,
this presentation reveals the monadic structure of $T$
immediately. Meanwhile, the structural equivalence relation can be
seen as the data required to present an algebra of the monad, $T$,
illustrating that an algebra of a monad, in this case $(T,\scong)$,
can also be a monad, in this case the free monad on $n$ generators.

To construct a logic where a formula, say $\phi$, denotes the
collection of monoid expressions in $T[G]$ satisfying $\phi$, we need
to specify what \emph{kind} of collection. This is a particular
insistence on specificity arising from the computer scientist's desire
for effective presentations. To turn specifications into computations,
we need to know what kind of collection to use because different kinds
of collections enjoy different constraints, which affect how we might
recursively compute a particular collection from a specification of
its elements.

For instance lists are sensitive to order where sets are not. Bags are
sensitive to multiplicity where sets are not. In some sense, sets are
the most insensitive monadically presented notion of collection, and
thus are ideal for erasing detail about collectivity that might
distract when bootstrapping an understanding of what it means to
collect or gather things together. It may be that this insensitivity
is why sets have enjoyed such a central role in the foundations of
mathematics.

In this context, it is irresistible to observe that collectivity is a
kind of composition, and thus, it is no accident that set theory and
category theory enjoy the relationship that they do: both are
presentations of mathematics as built up from a notion of
composition. Set theory begins with the most basic of notions, pure
collection, erasing all other details or qualities of collecting
components together into a composition. Category theory generalizes
the notion of composition, providing a simpler rubric for a much
broader range of phenomena. To say more, however, would be too much of
a digression.

Instead, let's use this discussion to motivate the selection of the
set monad as our notion of collection, not only because it is quite
common for the semantics of formulae to be sets, but because set's
insensitivity allow us to focus on more important aspects of the
construction we are describing.

\begin{itemize}
\item collection constraints. The first block of formulae enjoy no
  structural constraints, but only collection constraints. As such,
  they should be quite familiar. They constitute the well known
  relationship between boolean algebras and sets as their models.
    \begin{equation*}
      \begin{aligned}
        \meaningof{\mathsf{true}} = T[G] \\
        \meaningof{\mathsf{\neg}\phi} = T[G] \backslash \meaningof{\phi} \\
        \meaningof{\phi \mathsf{\&} \psi} = \meaningof{\phi} \cap \meaningof{\psi} \\
      \end{aligned}
    \end{equation*}
  \item structural constraints. The second block of formulae enjoy no
    collection constraints, but only structural constraints. These
    will be familiar to those who have worked with separation logic,
    behavioral spatial logic, or linear logic.
    \begin{equation*}
      \begin{aligned}
        \meaningof{\mathsf{e}} = \{ t \in T[G] : t \scong e \} \\
        \meaningof{\mathsf{g}_i} = \{ t \in T[G] : t \scong g_i \}  \\
        \meaningof{\phi \mathsf{*} \psi} = \{ t \in T[G] : t \scong t_1 * t_2, t_1 \in \meaningof{\phi}, \meaningof{\psi} \} \\
      \end{aligned}
    \end{equation*}
\end{itemize}


\subsubsection{Some formulae in a logic for monoids}

While this might seem a regular construction, does it yield
interesting logics? Indeed it does. Already in this simple logic we
can write down a 1-line formula for primality.

\begin{equation*}
  prime = \mathsf{\neg}\mathsf{e} \mathsf{\&} \mathsf{\neg}(\mathsf{\neg}\mathsf{e} \mathsf{\&} \mathsf{\neg}\mathsf{e})
\end{equation*}

It is easy to verify that 

\begin{equation*}
  \meaningof{prime} = \bigcup_i \meaningof{g_i}
\end{equation*}

\subsubsection{Varying the collection monad}

TBD

\subsection{A logic for lambda}

TBD

\subsubsection{Some formulae in a logic for lambda}

TBD

\subsection{A categorical rewrite}

In the categorical presentation of the term calculus for monoids,
we use a Lawvere 2-theory \cite{PowerLack, Yanofsky}.  We have a 
sort for terms, morphisms for the term constructors,
rewrite rules for structural equivalence and reduction, and relations
between rewrites.

In particular, we have a Lawvere 2-theory Th(Mon) with one sort 
$T$ for terms; the function symbols
\begin{grammar}
{T[G]} \bc e & \mbox{$e : 1 \to T$} \\
\;\;\; \bm \; g_i & \mbox{$g_i : 1 \to T$} \\
\;\;\; \bm \; T[G] * T[G] & \mbox{$m: T \times T \to T$,} \\
\end{grammar}
corresponding to the term constructors generate a $(G+1)$-pointed magma.

Structural equivalence is interpreted as bi-directional morphisms
$T \to T$. Most programming languages have free monoids built into
the language in the form of lists; some though, like Scheme and Haskell,
use an operator called ``cons'' to build lists out of ordered pairs,
exactly as we are doing with $m$.  In order to compare two
words of a monoid using the constructors above, we have to compute
a normal form.

We arbitrarily choose to have the parentheses at the right 
and no occurrences of $e()$ unless the normal form is $e()$ itself.  
Our rewrite rules are
$$\begin{array}{rll}
a:& ((x\;y)\;z) \Rightarrow (x\;(y\;z))\\
l:& (e()\;x) \Rightarrow x\\
r:& (x\;e()) \Rightarrow x\\
\end{array}$$
Finally, we impose relations on the rewrite rules.  The rewrites
$a,l,$ and $r$ satisfy the pentagon and triangle equations
from a monoidal category.

We can take models in any 2-category with finite products; if we take
models in the category Set thought of as a 2-category, then the
rewrite rules are identities, thus invertible, and we get
monoids generated by at least $G$. If instead we take models in Cat,
we get a category $T$ of words in a $(G+1)$-pointed magma with rewrites 
between them.  Once we begin to consider words with rewrites between
them, we call the words ``processes''.


\subsection{A logic for {\pic}}

\subsubsection{Our running process calculus}

\subsubsection{Syntax}
\label{syntax}
\begin{grammar}
{P} \bc \pzero & \mbox{stopped process} \\
       \;\;\; \bm \; {x}{?}{( y_1, \ldots, y_n )} \Rightarrow {P} & \mbox{input} \\
       \;\;\; \bm \; {x}{!}{( y_1, \ldots, y_n )} & \mbox{output} \\
%       \;\;\; \bm \; {M}{+}{N} & \mbox{choice} \\
%{ P, Q } \bc M & \mbox{include IO processes} \\                                
       \;\;\; \bm \; (\mathsf{new}\; x){P} & \mbox{new channel} \\
       \;\;\; \bm \; {P} \juxtap {Q} & \mbox{parallel} \\                                
\end{grammar}

Due to space limitations we do not treat replication, $!P$.

\subsubsection{Free and bound names}

\begin{equation*}
  \begin{aligned}
    & \freenames{\pzero} \defneqls \emptyset \\
    & \freenames{{x}{?}{( y_1, \ldots, y_n )} \Rightarrow {P}} \defneqls \\
    & \;\;\;\;\;\{ x \} \cup (\freenames{P} \setminus \{ y_1, \ldots y_n \}) \\
    & \freenames{{x}{!}{( y_1, \ldots, y_n )}} \defneqls \{ x, y_1, \ldots, y_n \} \\
    & \freenames{(\mathsf{new}\; x){P}} \defneqls \freenames{P} \setminus \{x\} \\    
    & \freenames{{P} \juxtap {Q}} \defneqls \freenames{P} \cup \freenames{Q} \\
  \end{aligned}
\end{equation*}

An occurrence of $x$ in a process $P$ is \textit{bound} if it is not
free. The set of names occurring in a process (bound or free) is
denoted by $\names{P}$.

\subsubsection{Structural congruence}
\label{congruence}

The {\em structural congruence} of processes, noted $\scong$, is the
least congruence containing $\alpha$-equivalence, $\alphaeq$, making
$( P, |, 0 )$ into commutative monoids and satisfying

\begin{equation*}  
  (\mathsf{new}\; x)(\mathsf{new}\; x){P} \scong (\mathsf{new}\; x)P
\end{equation*}
\begin{equation*}  
  (\mathsf{new}\; x)(\mathsf{new}\; y){P} \scong (\mathsf{new}\; y)(\mathsf{new}\; x)P
\end{equation*}
\begin{equation*}  
  ((\mathsf{new}\; x){P}) \juxtap {Q} \scong (\mathsf{new}\; x)({P} \juxtap {Q})
\end{equation*}

\subsubsection{Operational Semantics}\label{section:opsem}
 
\infrule[Comm]
{ |\vec{y}| = |\vec{z}| }
%{P_1 + {{ x_{0}{?}{(}{\vec{y}}{)} \concat {P}}\juxtap {x_{1}}{!}{(}{\vec{z}}{)} + P_2}
{{{ x{?}{(}{\vec{y}}{)} \concat {P}}\juxtap {x}{!}{(}{\vec{z}}{)}}
\red {{P}{\{}\vec{z}{/}{\vec{y}}{\}}}}

In addition, we have the following context rules:

\infrule[Par]{{P} \red {P}'}{{{P} \juxtap {Q}} \red {{P}' \juxtap {Q}}}

\infrule[New]{{P} \red {P}'}{{(\mathsf{new}\; x){P}} \red {(\mathsf{new}\; x){P}'}}

\infrule[Equiv]{{{P} \scong {P}'} \andalso {{P}' \red {Q}'} \andalso {{Q}' \scong {Q}}}{{P} \red {Q}}

\subsubsection{Bisimulation}

\begin{defn}
An \emph{observation relation}, $\downarrow$ is the smallest relation satisfying the rules
below.

\infrule[Out-barb]{ }
      {{x}!(\vec{y}) \downarrow x}
\infrule[Par-barb]{\mbox{$P\downarrow x$ or $Q\downarrow x$}}
      {\mbox{$P \juxtap Q \downarrow x$}}
\infrule[New-barb]{\mbox{$P\downarrow x$, $x \neq u$}}
      {\mbox{$(\mathsf{new}\; u){P} \downarrow x$}}

% We write $P \Downarrow x$ if there is $Q$ such that 
% $P \wred Q$ and $Q \downarrow x$.
\end{defn}

Notice that $\prefix{x}{y}{P}$ has no barb.  Indeed, in {\pic} as well
as other asynchronous calculi, an observer has no direct means to
detect if a sent message has been received or not.

\begin{defn}
%\label{def.bbisim}
An \emph{barbed bisimulation}, is a symmetric binary relation 
${\mathcal S}$ between agents such that $P\rel{S}Q$ implies:
\begin{enumerate}
\item If $P \red P'$ then $Q \red Q'$ and $P'\rel{S} Q'$.
\item If $P\downarrow x$, then $Q\downarrow x$.
\end{enumerate}
$P$ is barbed bisimilar to $Q$, written
$P \wbbisim Q$, if $P \rel{S} Q$ for some barbed bisimulation ${\mathcal S}$.
\end{defn}

\section{Logic as distributive law}

TBD

\section{Conclusions and future work}

TBD

\paragraph{Acknowledgments}

TBD

% ------------------------------------------------------------------------
%GATHER{Xbib.bib}   % For Gather Purpose Only
%GATHER{Paper.bbl}  % For Gather Purpose Only
\bibliographystyle{amsplain}
\bibliography{ladl}

% ------------------------------------------------------------------------



% ------------------------------------------------------------------------

\end{document}
% ------------------------------------------------------------------------
